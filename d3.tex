%%%%%%%%%%%%%%%%%%%%%%%%%%%%%%%%%%%%%%%%%%%%%%%%%%%%%%%%%%%%%%%%%%%%%%%%%%%%%%

\documentclass{l3deliverable}

%%%%%%%%%%%%%%%%%%%%%%%%%%%%%%%%%%%%%%%%%%%%%%%%%%%%%%%%%%%%%%%%%%%%%%%%%%%%%%

\usepackage{graphicx}%
%

\version{SVN Revision 1~ \

Made 28/10/2012~ by Team W}

\usepackage{tabularx}%
\usepackage{url}%
\usepackage{usecasedescription}%

%%%%%%%%%%%%%%%%%%%%%%%%%%%%%%%%%%%%%%%%%%%%%%%%%%%%%%%%%%%%%%%%%%%%%%%%%%%%%%
%% Check these macro values for appropriateness for your own document.

\title{Requirements Document}

\author{
    Gordon Reid: 1002536R\\
    Ryan Wells: 1002253W\\
    Kristopher Stewart: 1007175S\\
    David Selkirk: 1003646S\\
    James Gallagher: 0800899G\\
}

\date{29 October 2012}

\deliverableID{D3}
\project{PSD3 Group Exercise 1}
\team{W}

%%%%%%%%%%%%%%%%%%%%%%%%%%%%%%%%%%%%%%%%%%%%%%%%%%%%%%%%%%%%%%%%%%%%%%%%%%%%%%

\begin{document}

%%%%%%%%%%%%%%%%%%%%%%%%%%%%%%%%%%%%%%%%%%%%%%%%%%%%%%%%%%%%%%%%%%%%%%%%%%%%%%

\maketitle

\tableofcontents

\newpage

%%%%%%%%%%%%%%%%%%%%%%%%%%%%%%%%%%%%%%%%%%%%%%%%%%%%%%%%%%%%%%%%%%%%%%%%%%%%%%
%% Standard section for all documents

\section{Introduction}

\subsection{Identification}

\subsection{Related Documentation}

\subsection{Purpose and Description of Document}

\subsection{Document Status and Schedule}

%%%%%%%%%%%%%%%%%%%%%%%%%%%%%%%%%%%%%%%%%%%%%%%%%%%%%%%%%%%%%%%%%%%%%%%%%%%%%%

\section{Extended Problem Definition}

%%%%%%%%%%%%%%%%%%%%%%%%%%%%%%%%%%%%%%%%%%%%%%%%%%%%%%%%%%%%%%%%%%%%%%%%%%%%%%

The School of Computing Science are looking for a unified system for 
collecting, reviewing, and publishing internship advertisements. There are 
some main features which the system must possess: 

Submission of internship advertisements;

Review, comment, and publication of the internship advertisements;

Viewing of internship advertisements;

Notification of successful internship applications.
\\\\
The system does not have to support the actual application process, students
are to use the companies own channels for this.
\\\\
Companies wishing to submit advertisements to the system are required to first
contact the course coordinator for access to the system. This ensures no fake
or spam advertisements can be submitted.
\\\\
Only textual data can be submitted as part of an advertisement. There are no
strict guidelines for advertisement content so as a result, no automatic checks
on the content can be done by the system, only manually by the course
coordinator. The only possible check may be to ensure no duplicate entries
to the system, this is because companies with multiple vacancies for the same
job are only allowed a single advertisement which states the number of
available positions.
\\\\
Each advertisement has to be reviewed and accepted by the course coordinator
prior to submission for students to view. As part of the review process, the
course coordinator can comment on the advertisement, either for another member
of staff to look at prior to submission or as part of feedback to be sent back
to the company. In the event of an application being rejected, the system is
not required to submit feedback to the company, this is to be done separately
by the course coordinator.
\\\\
When an internship advertisement has been reviewed and published, students 
are to be notified somehow. This can be done either by mass email or via a 
notification presented to the user on login to the system.
\\\\
The system is available for all Computing Science students to access and 
they are all free to apply for any internship available. The login process could
conveniently be linked with existing student accounts and thus be handled separately
to the system.
\\\\
Students will have a status to allow the course coordinator to track progress
through the system. A student can either be accepted into an internship, pending
approval/response to an internship application, or yet to apply/no current 
application in progress.

%%%%%%%%%%%%%%%%%%%%%%%%%%%%%%%%%%%%%%%%%%%%%%%%%%%%%%%%%%%%%%%%%%%%%%%%%%%%%%

\section{System Scope}

Give an overview of the system here, in the context of the surrounding
environment.  Use case diagrams can be used to illustrate the
interactions between actors in the environment and the system.

You should explain the assumptions you have made in defining the
boundary of the system (i.e. what the system will and will not do).

Describe any conflicts in requirements expressed by different
stakeholders, how you resolved them and why.

%%%%%%%%%%%%%%%%%%%%%%%%%%%%%%%%%%%%%%%%%%%%%%%%%%%%%%%%%%%%%%%%%%%%%%%%%%%%%%

\subsection{System Actors}

Give descriptions of each of the actors that you have identified as
interacting with the system.

%%%%%%%%%%%%%%%%%%%%%%%%%%%%%%%%%%%%%%%%%%%%%%%%%%%%%%%%%%%%%%%%%%%%%%%%%%%%%%

\subsection{Domain Model}

Explain the elements of the domain here.

%%%%%%%%%%%%%%%%%%%%%%%%%%%%%%%%%%%%%%%%%%%%%%%%%%%%%%%%%%%%%%%%%%%%%%%%%%%%%%

\section{Use Case Descriptions}
This is a collection of use case descriptions (one per use case).
Think carefully about how to group these descriptions in the document.
You can use the template style provided to format your descriptions:

ADDED THIS TO CHECK


\begin{UseCaseTemplate}
\UseCaseLabel{Student Logs into system}
\UseCaseDescription{A student from the University of Glasgow
  interested in applying for a summer placement/internship gains
  access to the system by logging in.}
\UseCaseRationale{A student should be able to log into the
  system. This gives the system added protection and privacy to those
  who should not have access to the system.}
\UseCasePriority{High}
\UseCaseStatus{Uncomplete}
\UseCaseActors{Student}
\UseCaseExtensions{}
\UseCaseIncludes{}
\UseCaseConditions{}
\UseCaseNonFunctionalRequirements{}
\UseCaseScenarios{Successful log in, Unsuccessful log in.}
\UseCaseRisks{Unable to link with GUID log in which may mean students
  have to enrol for service. }
\UseCaseUserInterface{Log in screen.}
\end{UseCaseTemplate}

\begin{UseCaseTemplate}
\UseCaseLabel{Notification of new placement}
\UseCaseDescription{When a new placement is accepted by the course
  coordinator, all student in Level 3 CS should be notified of it by 
  email.}
\UseCaseRationale{A useful feature to aid CS students in their search
  for a placement}
\UseCasePriority{Medium}
\UseCaseStatus{Uncomplete}
\UseCaseActors{Student, Course Coordinator}
\UseCaseExtensions{}
\UseCaseIncludes{}
\UseCaseConditions{}
\UseCaseNonFunctionalRequirements{}
\UseCaseScenarios{Notification Sent}
\UseCaseRisks{Handeling the response from an email being sent to a 
  non-existent email address or to an email address that causes a 
  bounceback/send fail message. A student neglecting to view emails
  sent by the service due to mass emails sent when many advertisements
  go live on one day.}
\UseCaseUserInterface{Course Coordinator Advert View Screen.}
\end{UseCaseTemplate}

\begin{UseCaseTemplate}
\UseCaseLabel{}
\UseCaseDescription{}
\UseCaseRationale{}
\UseCasePriority{}
\UseCaseStatus{}
\UseCaseActors{}
\UseCaseExtensions{}
\UseCaseIncludes{}
\UseCaseConditions{}
\UseCaseNonFunctionalRequirements{}
\UseCaseScenarios{}
\UseCaseRisks{}
\UseCaseUserInterface{}
\end{UseCaseTemplate}


%%%%%%%%%%%%%%%%%%%%%%%%%%%%%%%%%%%%%%%%%%%%%%%%%%%%%%%%%%%%%%%%%%%%%%%%%%%%%%

\section{Non Functional Requirements}

Describe the non-functional requirements for the system here, giving a
rationale (traceable to your requirements gathering) for each.  You
will need to think about how to group/structure requirements in this
section.

%%%%%%%%%%%%%%%%%%%%%%%%%%%%%%%%%%%%%%%%%%%%%%%%%%%%%%%%%%%%%%%%%%%%%%%%%%%%%%

\section{Summary}

Give a (very short) summary of the key aspects of the requirements
specification.

%%%%%%%%%%%%%%%%%%%%%%%%%%%%%%%%%%%%%%%%%%%%%%%%%%%%%%%%%%%%%%%%%%%%%%%%%%%%%%

\appendix

Some suggested appendices are included below.

Appendices should be used to include information not completely
necessary to the understanding of the main document.

\section{Glossary}

Definitions.

\section{Scenarios}

A collection of scenarios you developed to exercise and refine your
use cases.

\section{Stakeholder Interview Documentation}

Any evidence you gathered from stakeholders relevant to your
requirements description.  You don't need to include everything
verbatim here, but summary documents, for example, identifying the key
points you identified (particularly if they relate to requirements
conflicts) can be useful.

\section{Stakeholder Panel Documentation}

(see above)

%%%%%%%%%%%%%%%%%%%%%%%%%%%%%%%%%%%%%%%%%%%%%%%%%%%%%%%%%%%%%%%%%%%%%%%%%%%%%%

\end{document}

%%%%%%%%%%%%%%%%%%%%%%%%%%%%%%%%%%%%%%%%%%%%%%%%%%%%%%%%%%%%%%%%%%%%%%%%%%%%%%